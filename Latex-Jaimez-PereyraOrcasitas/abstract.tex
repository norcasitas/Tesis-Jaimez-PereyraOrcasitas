%\begin{abstract}
\chapter*{Resumen}

\pagenumbering{roman} % para comenzar la numeracion de paginas en numeros romanos
\addcontentsline{toc}{chapter}{Resumen} % si queremos que aparezca en el índice

La tecnología de Web Service (WS) permite invocar aplicaciones externas, por ejemplo desde un motor de workflow y proporcionar beneficios significativos en el Workflow Management System (WFMS). Con el gran número de WS's que proporcionan una funcionalidad similar, es importante encontrar el mejor WS que cumpla con las necesidades del usuario, incluyendo tanto sus requisitos funcionales como los no funcionales. 
La descripción no funcional del Servicio requiere especificar en tiempo de ejecución los atributos de calidad que pueden influir en la elección de un WS ofrecido por un proveedor. En este sentido, es fundamental el uso de métricas para evaluar las características de calidad de los WS's (QoS) con el fin de filtrar los WS's conocidos y obtener el más adecuado.\\
El comportamiento dinámico de los WS’s en relación con el desarrollo de nuevos servicios y el constante cambio de los existentes requiere un proceso de evaluación continuo, que conduce a capturar información del WS con respecto a su calidad y rendimiento de evaluación conforme a lo solicitado por el workflow. Este trabajo propone, por un lado, refinar el proceso de contrastación de WS ofrecidos por proveedores y solicitados por usuarios del workflow, y por otro lado, definir y usar métricas para evaluar los atributos de calidad de dichos servicios.\\

\emph{Palabras clave}: Web Service o WS, workflow, características de calidad o QoS

\thispagestyle{empty} % para que no se numere esta página
%\end{abstract}

\chapter*{Agradecimientos}
\addcontentsline{toc}{chapter}{Agradecimientos} % si queremos que aparezca en el índice

{\sl Queremos agradecer a todas las personas que han hecho posible la realización de esta tesis. A nuestros directores de tesis Marcela Daniele y Ariel Gonzalez, que ha permitido llevar a un final exitoso este trabajo, por su asesoría, dedicación y sus valiosos aportes guiándonos sin ser directivas y mostrando en cada momento una inmejorable disposicón ante las dudas que durante la realización del trabajo nos surgieron, aportando siempre valiosas observaciones que tutelaron esta investigación.}\\
{\sl A nuestras familias y amigos, por su incondicional apoyo durante este largo camino.}\\
{\sl Además, debemos retribuir con cordial gratitud tanto a la Universidad Nacional de Río Cuarto como a la Facultad de Ciencias Exactas, Físico-Químicas y Naturales, y en especial al Departamento de Computación, por permitirnos desarrollar nuestras formaciones individuales de grado con excelente material académico, pero sobre todo humano, durante todos estos años.
}\\





%\end{flushright}